%-------------------------
% Resume in Latex
% Author : Jake Gutierrez
% Based off of: https://github.com/sb2nov/resume
% License : MIT
%------------------------

\documentclass[letterpaper,11pt]{article}

\usepackage{latexsym}
\usepackage[empty]{fullpage}
\usepackage{titlesec}
\usepackage{marvosym}
\usepackage[usenames,dvipsnames]{color}
\usepackage{verbatim}
\usepackage{enumitem}
\usepackage[hidelinks]{hyperref}
\usepackage{fancyhdr}
\usepackage[english]{babel}
\usepackage{tabularx}
\usepackage{fontawesome5}
\usepackage{multicol}
\usepackage[normalem]{ulem} % put this in the preamble
\setlength{\multicolsep}{-3.0pt}
\setlength{\columnsep}{-1pt}
\input{glyphtounicode}


%----------FONT OPTIONS----------
% sans-serif
% \usepackage[sfdefault]{FiraSans}
% \usepackage[sfdefault]{roboto}
% \usepackage[sfdefault]{noto-sans}
% \usepackage[default]{sourcesanspro}

% serif
% \usepackage{CormorantGaramond}
% \usepackage{charter}


\pagestyle{fancy}
\fancyhf{} % clear all header and footer fields
\fancyfoot{}
\renewcommand{\headrulewidth}{0pt}
\renewcommand{\footrulewidth}{0pt}

% Adjust margins
\addtolength{\oddsidemargin}{-0.6in}
\addtolength{\evensidemargin}{-0.5in}
\addtolength{\textwidth}{1.19in}
\addtolength{\topmargin}{-.5in}
\addtolength{\textheight}{1.4in}

\urlstyle{same}

\raggedbottom
\raggedright
\setlength{\tabcolsep}{0in}

% Sections formatting
\titleformat{\section}{
  \vspace{-4pt}\scshape\raggedright\large\bfseries
}{}{0em}{}[\color{black}\titlerule \vspace{-5pt}]

% Ensure that generate pdf is machine readable/ATS parsable
\pdfgentounicode=1

%-------------------------
% Custom commands
\newcommand{\resumeItem}[1]{
  \item\small{
    {#1 \vspace{-2pt}}
  }
}

\newcommand{\classesList}[4]{
    \item\small{
        {#1 #2 #3 #4 \vspace{-2pt}}
  }
}

\newcommand{\resumeSubheading}[4]{
  \vspace{-2pt}\item
    \begin{tabular*}{1.0\textwidth}[t]{l@{\extracolsep{\fill}}r}
      \textbf{#1} & \textbf{\small #2} \\
      \textit{\small#3} & \textit{\small #4} \\
    \end{tabular*}\vspace{-7pt}
}

\newcommand{\education}[6]{
  \vspace{-2pt}\item
    \begin{tabular*}{1.0\textwidth}[t]{l@{\extracolsep{\fill}}r}
      \textbf{#1} & \textbf{\small #2} \\
      \textit{\small#3} & \textit{\small #4} \\
    \end{tabular*}\vspace{-7pt}
}

\newcommand{\resumeSubSubheading}[2]{
    \item
    \begin{tabular*}{0.97\textwidth}{l@{\extracolsep{\fill}}r}
      \textit{\small#1} & \textit{\small #2} \\
    \end{tabular*}\vspace{-7pt}
}

\newcommand{\resumeProjectHeading}[2]{
    \item
    \begin{tabular*}{1.001\textwidth}{l@{\extracolsep{\fill}}r}
      \small#1 & \textbf{\small #2}\\
    \end{tabular*}\vspace{-7pt}
}

\newcommand{\resumeSubItem}[1]{\resumeItem{#1}\vspace{-4pt}}

\renewcommand\labelitemi{$\vcenter{\hbox{\tiny$\bullet$}}$}
\renewcommand\labelitemii{$\vcenter{\hbox{\tiny$\bullet$}}$}

\newcommand{\resumeSubHeadingListStart}{\begin{itemize}[leftmargin=0.0in, label={}]}
\newcommand{\resumeSubHeadingListEnd}{\end{itemize}}
\newcommand{\resumeItemListStart}{\begin{itemize}}
\newcommand{\resumeItemListEnd}{\end{itemize}\vspace{-5pt}}

%-------------------------------------------
%%%%%%  RESUME STARTS HERE  %%%%%%%%%%%%%%%%%%%%%%%%%%%%
\begin{document}

%----------HEADING----------
\begin{center}
    {\Huge \textbf{Vedant Borkute}} \\[6pt]
    \small
    \raisebox{-0.1\height}\faPhone\ +1 951-823-2856 \quad
    \href{mailto:2711vedant@gmail.com}{\raisebox{-0.2\height}\faEnvelope\ \uline{2711vedant@gmail.com}} \quad
    \href{mailto:vbork001@ucr.edu}{\raisebox{-0.2\height}\faGlobe\ \uline{vbork001@ucr.edu}} \quad
    \href{https://github.com/Vedant2100}{\raisebox{-0.2\height}\faGithub\ \uline{Vedant2100}} \quad
    \href{https://www.linkedin.com/in/vedant-b-1290a51a6/}{\raisebox{-0.2\height}\faLinkedin\ \uline{vedant-borkute}}
\end{center}

%-----------EDUCATION-----------
\section{Education}
\noindent\begin{tabular*}{\textwidth}{@{\extracolsep{\fill}} l r}
  \textbf{University of California, Riverside} & \textit{Sep 2025 -- Mar 2027} \\
  \textit{MS in Computer Science} & Riverside, California \\
\end{tabular*}

\vspace{6pt}

\noindent\begin{tabular*}{\textwidth}{@{\extracolsep{\fill}} l r}
  \textbf{Indian Institute of Technology, Bombay} & \textit{Aug 2019 -- Apr 2023} \\
  \textit{B.Tech in Computer Science} & Mumbai, Maharashtra, India \\
\end{tabular*}
\vspace{-5pt}
\vspace{-10pt}
%-----------EXPERIENCE-----------
\section{Experience}
\textbf{Data Scientist} - \textit{Finarb.ai}\hfill \textit{Aug 2023 - Aug 2025}\\

\vspace{-8pt}
\resumeItemListStart
\resumeItem{Implemented and benchmarked forecasting statistical(Seasonal AR, VAR, Facebook Prophet) and deep learning based models (LSTM, TCN) for order data across different nodes in pharmaceutical supply chains, achieving a \textbf{90\% reduction in Mean Absolute Percentage Error}.}
\vspace{-3pt}
    \resumeItem{Conducted time-series diagnostics \textbf{ACF/PACF plots \& EACF matrices, mean and variance stationarity tests}, and incorporated \textbf{exogenous variables} like mortality rates, veteran population statistics, inventory, backorders) into forecasting models to improve forecast accuracy.}
\vspace{-3pt}
    \resumeItem{Built an \textbf{end-to-end forecasting pipeline} including data generation, pre-processing, model selection with custom metrics, and docker-based containerization for reproducible deployment.}
    \vspace{-3pt}
    \resumeItem{Implemented AI Agents: 
    \vspace{-3pt}
      \begin{itemize}[leftmargin=*,label=--,noitemsep,nolistsep]
        \item \textbf{Feature Engineering:} Generation of domain-relevant features via retrieval-augmented generation on internet-based and other sources, improving model performance by 5\%..
        \item \textbf{Cross-Database Queries:} Enabling natural language queries spanning multiple databases/datasources with a SQL and Python code generation agentic workflow.
        \item \textbf{Business Dashboards:} Automated creation of dashboards with key performance indicators and intuitive visualizations.
      \end{itemize}
    }
    \vspace{-3pt}
    \resumeItem{Performed data preparation and analysis for pharmaceutical use cases (inventory-order correlations, order-pricing correlations, supply scenario assessment, purchasing pattern) using optimized \textbf{SQL scripts} to generate datasets, compute metrics, and present summaries to leadership.}
    \vspace{-3pt}\resumeItem{Developed comprehensive \textbf{Power BI reports} for share gain attribution, portfolio performance, competitor dynamics (top share gaps, vulnerability trends, consecutive share losses), BCG-style prioritization, and annual revenue/volume dashboards with executive summaries.}
  \resumeItemListEnd
\vspace{-5pt}
\vspace{-8pt}
%-----------PROJECTS-----------
\section{Projects}
\vspace{-8pt}
\resumeSubHeadingListStart
  \resumeProjectHeading
      {\textbf{Online Competing and Development Environment} \textit{-- Advisor: Prof. Amitabha Sanyal}}{Autumn 2020}
      \resumeItemListStart
        \resumeItem{Built and deployed a remote virtual Linux environment to eliminate inter-OS compatibility issues and enable cross-platform code compilation via a web browser.}
        \resumeItem{Developed a \textbf{Django backend} supporting user accounts and storage of up to 10 files with directory structure; designed the frontend using \textbf{Bootstrap}.}
        \resumeItem{Added support for multiple programming languages (C++, Java, Python) with standard library support.}
        \resumeItem{Implemented an online competing environment where users could attempt scheduled problems, with submissions auto-graded based on input parameters.}
      \resumeItemListEnd
\resumeSubHeadingListEnd
\vspace{-5pt}
\vspace{-10pt}
%-----------PROGRAMMING SKILLS-----------
\section{Technical Skills}
\resumeItemListStart
    \resumeItem{\textbf{Languages:} C++, R, JavaScript, Python, Bash, Prolog, AWK, MIPS, Neo4J CQL, SQL, Sed, Lex, Yacc, \LaTeX}
    \vspace{-3pt}\resumeItem{\textbf{Technologies:} Pandas, R, SQL, Power BI, NumPy, SciPy, MATLAB, PostgreSQL, React, Django, PyTorch, TensorFlow, MongoDB}
    \vspace{-3pt}\resumeItem{\textbf{Software:} Microsoft Excel, RStudio, DataGrip, Cursor, Docker, Git, PowerPoint, Word, Slack, Azure DevOps, Jira, Confluence}
\resumeItemListEnd
\vspace{-5pt}
\vspace{-10pt}
%-----------ACCOMPLISHMENTS-----------
\section{Accomplishments}
\resumeItemListStart
    \resumeItem{Ranked AIR 1 (OBC-PwD) in IIT-JEE Advanced 2019, among the top 0.8\% nationwide in JEE Main among 1.3 million candidates.}
    % \resumeItem{Received the Late Kalpana Chawla Best Student Award by Shivaji Science Junior College for outstanding performance throughout Higher Secondary Education.}
\resumeItemListEnd
\end{document}




% \subsection*{Custom Linux Shell and Feature Extension of xv6 (Autumn 2021)} \hfill\\
% \textbf{Guide:} Prof. Mythili Vutukuru IIT Bombay \\
% • Built a shell in C++ capable of serial, parallel background execution of single or multiple Unix commands \\
% • Created custom signal handler to enable controlled termination of foreground processes or the shell itself \\
% • Examined xv6 source code and implemented a variant of fork system call and simple version of wait, exec system calls.

% \subsection*{Image Segmentation (Autumn 2020)} \hfill\\
% \textbf{Guide:} Prof. Amitabha Sanyal IIT Bombay \\
% • Used K-means++ algorithm and Python SciPy Library to smoothen sharp contrast images. \\
% • Utilized NumPy, Pandas, SciPy libraries of Python to store pixels of images in matrix form. \\
% • Used K-means algorithm to find centroid and replaced pixels by its centroid for image segmentation.

% \subsection*{COVID-19 Pandemic Predictions (Autumn 2020)} \hfill\\
% \textbf{Guide:} Prof. Amitabha Sanyal IIT Bombay \\
% • Applied Levitt’s H(t) on Indian COVID-19 death’s data read from a CSV file Using Pandas \\
% • Plotted a scatter plot of data using Matplotlib to visualize the behavior of Levitt’s H(t) Metric \\
% • Predicted the end of a pandemic using the Linear fit plot of data points by implementing Linear regression from Scipy module 
% • The linear extraplotation-based prediction is fairly effective, at least when values of H(t) are small.

% \subsection*{Morphism (Autumn 2020)} \hfill\\
% \textbf{Guide:} Prof. Ajit A Diwan IIT Bombay \\
% • Learned about Fibonacci sequence , thue-morse sequence and properties of their morphism. \\
% • Taking properties of morphism implemented functions to find the longest common sub-sequences and substring with help of Dynamic Programming and Knuth-Morris-Pratt algorithm.

% \subsection*{Movie Recommendation System (Autumn 2022)} \hfill\\
% \textbf{Guide:} Prof. Abir De IIT Bombay \\
% • Analyzed a set of users with their previous ratings for a set of movies and then predict the rating they will assign to a movie they have not previously rated. Learned about Content-based and Collaborative filtering.

    
% \vspace{-15pt}
% \section{Publications}
%  \begin{itemize}[leftmargin=0.15in, label={}]
%     \small{\item{P. Bhattacharya, K. Hiware, S. Rajgaria, \textbf{N. Pochhi}, K. Ghosh, S. Ghosh. \href{https://link.springer.com/chapter/10.1007/978-3-030-15712-8_27}{A Comparative Study of Summarization Algorithms applied to Legal Case Judgments}. In \emph{European Conference on Information Retrieval, 2019}
%     }}
%  \end{itemize}
%  \vspace{-12pt}
